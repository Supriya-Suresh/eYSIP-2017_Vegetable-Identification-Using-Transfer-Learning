\documentclass[a4paper,12pt,oneside]{book}

%-------------------------------Start of the Preable------------------------------------------------
\usepackage[english]{babel}
\usepackage{blindtext}
%packagr for hyperlinks
\usepackage{hyperref}
\hypersetup{
    colorlinks=true,
    linkcolor=blue,
    filecolor=magenta,      
    urlcolor=cyan,
}

\urlstyle{same}
%use of package fancy header
\usepackage{fancyhdr}
\setlength\headheight{26pt}
\fancyhf{}
%\rhead{\includegraphics[width=1cm]{logo}}
\lhead{\rightmark}
\rhead{\includegraphics[width=1cm]{logo.png}}
\fancyfoot[RE, RO]{\thepage}
\fancyfoot[CE, CO]{\href{http://www.e-yantra.org}{www.e-yantra.org}}

\pagestyle{fancy}

%use of package for section title formatting
\usepackage{titlesec}
\titleformat{\chapter}
  {\Large\bfseries} % format
  {}                % label
  {0pt}             % sep
  {\huge}           % before-code
 
%use of package tcolorbox for colorful textbox
\usepackage[most]{tcolorbox}
\tcbset{colback=cyan!5!white,colframe=cyan!75!black,halign title = flush center}

\newtcolorbox{mybox}[1]{colback=cyan!5!white,
colframe=cyan!75!black,fonttitle=\bfseries,
title=\textbf{\Large{#1}}}

%use of package marginnote for notes in margin
\usepackage{marginnote}

%use of packgage watermark for pages
%\usepackage{draftwatermark}
%\SetWatermarkText{\includegraphics{logo}}
\usepackage[scale=2,opacity=0.1,angle=0]{background}
\backgroundsetup{
contents={\includegraphics{logo}}
}

%use of newcommand for keywords color
\usepackage{xcolor}
\newcommand{\keyword}[1]{\textcolor{red}{\textbf{#1}}}

%package for inserting pictures
\usepackage{graphicx}

%package for highlighting
\usepackage{color,soul}

%new command for table
\newcommand{\head}[1]{\textnormal{\textbf{#1}}}


%----------------------End of the Preamble---------------------------------------


\begin{document}

%---------------------Title Page------------------------------------------------
\begin{titlepage}
\raggedright
{\Large eYSIP2017\\[1cm]}
{\Huge\scshape Vegetable Identification Using Transfer Learning \\[.1in]}
\vfill
\begin{flushright}
{\large Sanket Shanbhag \\}
{\large Supriya Suresh \\}
{\large Saurav, Naveen and Khalid \\}
{\large Duration of Internship: $ 22/05/2017-07/07/2017 $ \\}
\end{flushright}

{\itshape 2017, e-Yantra Publication}
\end{titlepage}
%-------------------------------------------------------------------------------

\chapter[Project Tag]{Vegetable Identification Using \\ Transfer Learning}
\section*{Abstract}
This project aims to create a system for automatically logging farm produce from the green house at e-Yantra. An image of the produce is captured and this image is fed to a neural network based on the Inception-v3 model by Google. This system then identifies the vegetable and classifies it into different classes it has been trained on. Also, a re-training system is implemented which automatically trains any new images captured and thereby further increasing the accuracy of the system over time.

\subsection*{Completion status}
The system has been successfully trained for 18 classes (vegetables) achieving accuracy of 99.5\% on a separate test set. Also a system has been implemented for auto-training any new images captured on the system every 4 days.

\section{Hardware parts}
\begin{itemize}
  \item List of hardware 
  \item Detail of each hardware: \href[page=5]{./datasheet/MPU-9150.pdf}{Datasheet, page 5}, \href{http://www.amazon.in}{Vendor link}, 
  \item Connection diagram
\end{itemize}
\newpage
\section{Software used}
\begin{itemize}
  \item \href{https://www.python.org/downloads/release/python-352/}{Python 3.5.2}
  \item \href{https://pypi.python.org/pypi/tensorflow/1.2.0}{Tensorflow 1.2.0}
  \item \href{https://github.com/google/prettytensor}{PrettyTensor 0.7.4} 
  \item Installation Steps:
  \begin{itemize}
  	\item Open a terminal in the project base folder and type: \\
  	\texttt{pip3 install -r requirements.txt}
  \end{itemize}
\end{itemize}

\section{Assembly of hardware}
Circuit diagram and Steps of assembly of hardware with pictures for each step
\subsection*{Circuit Diagram}
Circuit schematic, simplified circuit diagram , block diagram of system
\subsection*{Step 1}
Steps for assembling part 1
\subsection*{Step 2}
Steps for assembling part 2
\subsection*{Step 3}
Steps for assembling part 3



\section{Software and Code}
The complete code is available \href{https://github.com/eYSIP-2017/eYSIP-2017_Vegetable-Identification-Using-Transfer-Learning}{here}. It is divided into various folders each with a self contained module of the project.

\subsection{Downloading and formatting data}
The folder \texttt{download\_data} contains python scripts and shell scripts for downloading and formatting data from a list of URLS or from \href{http://www.image-net.org/}{ImageNet}. Information on using these scripts can be found in the \href{https://github.com/eYSIP-2017/eYSIP-2017_Vegetable-Identification-Using-Transfer-Learning/wiki/Downloading-and-formatting-data.}{wiki page}.

\subsection{Transfer Learning}
The code for transfer learning can be found in the \texttt{transfer\_on\_inception\_v3} folder. The \texttt{transferveg.py} file contains the code for running the model.
We remove the final layer of the Inception-v3 model and add 3 fully connected layers of 4096, 2048, and 1024 nodes with a dropout layer after the 2048-node layer.
This helps us achieve an accuracy of 99.5\%

\subsection{Module Integration}
The \texttt{ghfarm.py} script contained in the \texttt{module\_integration} folder will run on the raspberry pi on the weighing machine. It interfaces with the server to predict the crop and sends data once the image is taken.

\subsection{Auto-Training}
The \texttt{AutoTrain} folder contains code for running a server to accept images and scripts for adding cron tasks to run the autotraining code every 4 days.

\newpage
\section{Use and Demo}
Final Setup Image
%\begin{figure}[!ht]
%	\centering
%	\includegraphics[width=0.5\linewidth]{"../../Machine Learning Workshop/setup"}
%	\caption{Setup}
%	\label{fig:setup}
%\end{figure}


Instructions for using the weighing machine can be found \href{https://github.com/eYSIP-2017/eYSIP-2017_Vegetable-Identification-Using-Transfer-Learning/blob/master/Manual/WeighingMachineManual.pdf}{here}. And here is the \href{http://www.youtube.com}{Youtube Link} of the video demonstration.

 

\section{Future Work}
This project focussed mainly on vegetables, in future it can be extended to fruits as well resulting in many classes for Identification.
\newpage
\section{Bug report and Challenges}
**Any issues in code and hardware.**This is to be done.**\\\\
One of the challenges we faced was lack of proper data for training hence we collected live images from Vegetable Market.\\
Every Vegetable has different varities so our second challenge was to train our system to as many varities as possible.\\

**Or else we can write in bullet format**
\begin{itemize}
	\item \textbf{Challenge 1: }Lack of proper data for training hence collected live images of Vegetables from Market
	\item \textbf{Challenge 2: }Every Vegetable has its own different varities so our second challenge was to train our system to as many varities as possible.
\end{itemize}

\begin{thebibliography}{li}
\bibitem{wavelan97}
Ad Kamerman and Leo Monteban,
{\em WaveLAN-II: A High-Performance Wireless LAN for the Unlicensed band},
1997.

\end{thebibliography}


\end{document}

